\documentclass[12pt,a4paper]{article}

\usepackage[brazil]{babel}
\usepackage[utf8]{inputenc}
\usepackage[T1]{fontenc}
\usepackage{setspace}
\usepackage{geometry}
\usepackage{amsmath, amssymb}
\usepackage{graphicx}
\usepackage{booktabs}
\usepackage{multirow}

\geometry{
    left=3cm,
    right=2cm,
    top=3cm,
    bottom=2cm
}

\onehalfspacing

\begin{document}

\begin{center}
    \textbf{UNIVERSIDADE FEDERAL DO MARANHÃO} \\
    \textbf{CENTRO DE CIÊNCIAS EXATAS E TECNOLOGIA} \\[3cm]

    \textbf{DISCIPLINA:} Pesquisa Operacional \\
    \textbf{DOCENTE:} Alexandre Cesar Muniz de Oliveira \\
    \textbf{DISCENTE:} Renef Ricardo Costa da Silva \\[4cm]

    \textbf{\Large A Jornada dos Bacuris} \\[0.5cm]
    \textbf{Relatório Técnico de Otimização (MIP)} \\[4cm]

    São Luís -- MA \\
    2025
\end{center}

\newpage

\section{Introdução}

A otimização de sistemas logísticos com múltiplos estágios de transporte e armazenagem é
fundamental para a viabilidade econômica de cadeias produtivas complexas. Quando decisões
sobre a ativação de instalações intermediárias estão envolvidas, o problema assume caráter
estratégico e pode ser adequadamente modelado por meio da Programação Inteira Mista (MIP).

O presente trabalho aborda o Problema de Transbordo com Custos Fixos aplicado à cadeia
logística da produção de bacuri no Norte do Brasil. Este problema combina variáveis contínuas
(fluxos de produto) e variáveis discretas (abertura de silos e quantidade de veículos),
caracterizando um modelo MIP que determina simultaneamente rotas, pontos de transbordo ativos
e recursos de transporte necessários.

A relevância deste estudo estende-se além do caso específico, oferecendo metodologia aplicável
a diversas cadeias agroindustriais que operam com infraestrutura limitada e requerem
transporte multimodal.

\section{Contextualização do Problema}

\subsection{O Sistema Logístico}

A cadeia logística da produção de bacuri envolve múltiplas etapas de transporte devido às
grandes distâncias entre áreas de produção e centros exportadores:

\begin{enumerate}
    \item \textbf{Fazendas $\rightarrow$ Silos:} Treminhões (caminhões com até 3 carrocerias
    articuladas) transportam a produção por vias não pavimentadas até silos de transbordo.

    \item \textbf{Silos (Transbordo):} Instalações temporárias que armazenam os frutos por até
    72 horas. Podem ser ativadas mediante custo fixo.

    \item \textbf{Silos $\rightarrow$ Porto:} Locomotivas realizam transporte de longa distância
    com custo proporcional por tonelada.

    \item \textbf{Porto $\rightarrow$ Exportação:} Produto embarcado em navios para mercados
    internacionais.
\end{enumerate}

\subsection{Características do Problema}

\begin{itemize}
    \item Múltiplos nós: fazendas (origens), silos (intermediários), portos (destinos)
    \item Múltiplos modais: treminhões, trem, navio
    \item Custos fixos: ativação de silos e uso de carrocerias
    \item Capacidades limitadas: treminhões e silos
    \item Conservação de fluxo nos nós intermediários
    \item Ativação condicional de portos (lote mínimo)
\end{itemize}

\subsection{Justificativa dos Silos}

Os silos permitem: consolidação de carga de múltiplas fazendas, compatibilidade entre modais
incompatíveis (treminhão ↔ trem), flexibilidade temporal e redução de custos versus transporte
direto.

\section{Objetivos}

\subsection{Objetivo Geral}

Desenvolver um modelo MIP que minimize o custo total da operação logística, determinando
fluxos de produto, ativação de silos e quantidade de carrocerias de treminhão.

\subsection{Objetivos Específicos}

\begin{enumerate}
    \item Mapear o sistema logístico (nós, arcos, capacidades, custos)
    \item Formular matematicamente o problema (conjuntos, parâmetros, variáveis, FO, restrições)
    \item Implementar computacionalmente em Pyomo
    \item Criar e resolver três instâncias (A: baixo custo fixo, B: alto custo fixo, C: referência)
    \item Analisar comparativamente os resultados
    \item Discutir aplicabilidade, limitações e extensões
\end{enumerate}

\section{Descrição Textual do Sistema Logístico}

\subsection{Nós da Rede}

\textbf{Origens:} $F1, F2, F3, F4, F5$ (fazendas produtoras)

\textbf{Intermediários:} $S1, S2, S3, S4$ (silos de transbordo)

\textbf{Destinos:} $P1, P2, P3$ (terminais portuários)

\subsection{Arcos e Modais}

\textbf{Fazenda $\rightarrow$ Silo (Treminhão)}
\begin{itemize}
    \item Arcos: $(F_i, S_j)$ para $i \in \{1,\dots,5\}$, $j \in \{1,\dots,4\}$
    \item Capacidade: 30t por carroceria (1 a 3 carrocerias)
    \item Custo: \textbf{apenas fixo} de \$20 por carroceria (sem custo variável por tonelada)
\end{itemize}

\textbf{Silo $\rightarrow$ Porto (Ferrovia)}
\begin{itemize}
    \item Arcos: $(S_j, P_k)$ para $j \in \{1,\dots,4\}$, $k \in \{1,\dots,3\}$
    \item Custo: proporcional (\$/t), variando por rota
\end{itemize}

\subsection{Capacidades}

\begin{center}
\begin{tabular}{|l|c|c|}
\hline
\textbf{Elemento} & \textbf{Capacidade} & \textbf{Unidade} \\
\hline
Carroceria & 30 & toneladas \\
Carrocerias/treminhão & 1 a 3 & unidades \\
Silo S1 & 130 & toneladas \\
Silo S2 & 120 & toneladas \\
Silo S3 & 140 & toneladas \\
Silo S4 & 160 & toneladas \\
\hline
\end{tabular}
\end{center}

\section{Modelagem Matemática}

\subsection{Conjuntos e Índices}

\begin{itemize}
    \item $F = \{F1, F2, F3, F4, F5\}$: fazendas
    \item $S = \{S1, S2, S3, S4\}$: silos
    \item $P = \{P1, P2, P3\}$: portos
    \item $i \in F$, $j \in S$, $k \in P$: índices
\end{itemize}

\subsection{Parâmetros}

\textbf{Produção e Demanda:}
\begin{align*}
\text{prod} &= \{F1: 120, F2: 80, F3: 95, F4: 110, F5: 130\} \text{ (t)} \\
\text{dem} &= \{P1: 200, P2: 220, P3: 215\} \text{ (t)}
\end{align*}

\textbf{Capacidades:}
\begin{align*}
\text{cap\_silo} &= \{S1: 130, S2: 120, S3: 140, S4: 160\} \text{ (t)} \\
\text{cap\_carr} &= 30 \text{ t/carroceria}
\end{align*}

\textbf{Custos:}
\begin{itemize}
    \item $\text{cf\_silo}_j$: custo fixo de ativação do silo $j$ (\$)
    \item $\text{cf\_carr} = 20$: custo fixo por carroceria (\$)
    \item $c_{jk}$: custo ferroviário S→P (\$/t), matriz $4 \times 3$
\end{itemize}

\subsection{Variáveis de Decisão}

\textbf{Contínuas:}
$x_{ij} \geq 0$ (fluxo F→S), $y_{jk} \geq 0$ (fluxo S→P)

\textbf{Binárias:}
$z_j \in \{0,1\}$ (ativação silo), $w_k \in \{0,1\}$ (ativação porto)

\textbf{Inteiras:}
$t_{ij} \in \{0,1,2,3\}$ (carrocerias F→S)

\subsection{Função Objetivo}

\[
\min Z =
\sum_{j \in S} \text{cf\_silo}_j \cdot z_j
+ \sum_{(i,j)} \text{cf\_carr} \cdot t_{ij}
+ \sum_{(j,k)} c_{jk} \cdot y_{jk}
\]

\subsection{Restrições}

\textbf{R1. Conservação de Fluxo nas Fazendas}
\[
\sum_{j \in S} x_{ij} = \text{prod}_i \quad \forall i \in F
\]

\textbf{R2. Conservação de Fluxo nos Silos}
\[
\sum_{i \in F} x_{ij} = \sum_{k \in P} y_{jk} \quad \forall j \in S
\]

\textbf{R3. Ativação Condicional dos Portos}
\[
\text{dem}_k \cdot w_k \leq \sum_{j \in S} y_{jk} \leq M \cdot w_k \quad \forall k \in P
\]

\textbf{R4. Capacidade dos Silos}
\[
\sum_{i \in F} x_{ij} \leq \text{cap}_j \cdot z_j \quad \forall j \in S
\]

\textbf{R5. Capacidade das Carrocerias}
\[
x_{ij} \leq 30 \cdot t_{ij} \quad \forall i \in F, j \in S
\]

\section{Metodologia Computacional}

\subsection{Ambiente e Ferramentas}

\begin{itemize}
    \item \textbf{Linguagem:} Python 3.10
    \item \textbf{Biblioteca de Modelagem:} Pyomo 6.4.2
    \item \textbf{Solver:} GLPK 5.0 (open-source)
    \item \textbf{Ambiente:} Local (Windows)
\end{itemize}

\subsection{Estrutura do Projeto}

\begin{verbatim}
transbordo/
├── modelo.py       # Modelo Pyomo
├── instancias.py   # Dados A, B, C
└── resolve.py      # Execução e resultados
\end{verbatim}

\subsection{Instâncias Criadas}

\textbf{Instância A (baixo custo fixo):}
$\text{cf\_silo} = \{S1: 100, S2: 90, S3: 110, S4: 95\}$

\textbf{Instância B (alto custo fixo):}
$\text{cf\_silo} = \{S1: 900, S2: 850, S3: 920, S4: 880\}$

\textbf{Instância C (referência):}
$\text{cf\_silo} = \{S1: 500, S2: 450, S3: 520, S4: 480\}$

\section{Resultados e Análise Crítica}

\subsection{Silos Ativados}

\textbf{Resultado:} Nas três instâncias, todos os 4 silos foram ativados.

\textbf{Justificativa:} A produção total (535t) excede a capacidade dos 3 maiores silos
combinados ($160 + 140 + 130 = 430$ t). Logo, S2 (120t) é estruturalmente necessário,
independentemente do custo fixo.

\textbf{Volumes processados por silo (idênticos nas três instâncias):}
\begin{itemize}
    \item S1: 115t / 130t (88,5\% de utilização)
    \item S2: 120t / 120t (SATURADO)
    \item S3: 140t / 140t (SATURADO)
    \item S4: 160t / 160t (SATURADO)
\end{itemize}

\subsection{Custos Totais e Decomposição}

\begin{table}[h]
\centering
\caption{Decomposição de custos por instância}
\label{tab:custos}
\begin{tabular}{lccc}
\toprule
\textbf{Componente (R\$)} & \textbf{Inst. A} & \textbf{Inst. C} & \textbf{Inst. B} \\
\midrule
Custo fixo silos & 395,00 & 1.950,00 & 3.550,00 \\
Custo carrocerias & 400,00 & 400,00 & 400,00 \\
Transporte S$\rightarrow$P (ferro) & 8.850,00 & 8.850,00 & 8.850,00 \\
\midrule
\textbf{TOTAL} & \textbf{9.645,00} & \textbf{11.200,00} & \textbf{12.800,00} \\
\bottomrule
\end{tabular}
\end{table}

\textit{Nota:} O custo do trecho Fazenda$\rightarrow$Silo está incluído no componente ``Custo
carrocerias'', pois não existe custo variável por tonelada neste trecho, apenas o custo fixo
de R\$ 20 por carroceria utilizada.

\subsection{Análise Percentual}

O transporte ferroviário (S→P) representa a maior parcela do custo total em todas as
instâncias:

\begin{itemize}
    \item \textbf{Instância A:} 91,8\% (ferrovia), 4,1\% (silos), 4,1\% (carrocerias)
    \item \textbf{Instância C:} 79,0\% (ferrovia), 17,4\% (silos), 3,6\% (carrocerias)
    \item \textbf{Instância B:} 69,1\% (ferrovia), 27,7\% (silos), 3,1\% (carrocerias)
\end{itemize}

O componente ferroviário domina o custo em todas as instâncias porque:
\begin{itemize}
    \item Concentra o maior volume transportado (535t)
    \item Possui matriz de custos diferenciados por rota
    \item É o único modal com custo variável por tonelada
\end{itemize}

\subsection{Comparação Relativa}

\begin{table}[h]
\centering
\caption{Comparação das três instâncias}
\label{tab:comparativa}
\begin{tabular}{lccc}
\toprule
\textbf{Indicador} & \textbf{Inst. A} & \textbf{Inst. C} & \textbf{Inst. B} \\
\midrule
Custo total (R\$) & 9.645,00 & 11.200,00 & 12.800,00 \\
Silos ativados & 4 & 4 & 4 \\
Custo fixo total (R\$) & 395,00 & 1.950,00 & 3.550,00 \\
Carrocerias utilizadas & 20 & 20 & 20 \\
Utilização agregada (\%) & 89,2 & 89,2 & 89,2 \\
\bottomrule
\end{tabular}
\end{table}

A Instância A é a mais barata (R\$ 9.645,00) devido ao baixo custo fixo dos silos. A Instância
C representa o cenário base (R\$ 11.200,00). A Instância B é a mais cara (R\$ 12.800,00),
com aumento de 32,7\% em relação à A.

\textbf{Diferenças percentuais:}
\begin{itemize}
    \item C vs A: R\$ 1.555,00 (+16,1\%)
    \item B vs C: R\$ 1.600,00 (+14,3\%)
    \item B vs A: R\$ 3.155,00 (+32,7\%)
\end{itemize}

\subsection{Sensibilidade ao Custo Fixo}

\textbf{Observação chave:} A variação nos custos fixos dos silos (de R\$ 395 na Instância A
para R\$ 3.550 na Instância B) \textbf{não altera a decisão estrutural} de quais silos
ativar (sempre 4), mas altera significativamente o custo total.

Isso ocorre porque o sistema opera \textbf{próximo ao limite de capacidade}, caracterizando
um ``efeito limiar'': aumentos no custo fixo encarecem a solução sem modificar a configuração.

\subsection{Distribuição dos Fluxos}

\textbf{Portos ativos:} Em todas as três instâncias, apenas o porto P3 foi ativado,
recebendo toda a produção de 535t (248,8\% de sua capacidade nominal de 215t).

Este resultado indica que:
\begin{itemize}
    \item P3 possui as rotas ferroviárias mais econômicas (S4→P3 e S3→P3 custam apenas R\$ 10/t)
    \item A restrição condicional de porto permite operação acima da capacidade nominal
    \item P1 e P2 não foram ativados por não atingirem a demanda mínima com a distribuição ótima
\end{itemize}

\subsection{Impacto dos Modais}

\subsubsection{Modal Rodoviário (Treminhão)}

\textbf{Características:}
\begin{itemize}
    \item Custo total: R\$ 400,00 (constante nas três instâncias)
    \item Carrocerias utilizadas: 20 unidades
    \item Capacidade contratada: 600t (20 × 30t)
    \item Volume transportado: 535t
    \item Utilização agregada: 89,2\%
\end{itemize}

\textbf{Rotas no limite:} Duas rotas operam com 3 carrocerias (limite máximo): F1→S3 e
F5→S4, indicando que a restrição operacional está ativa nesses pontos.

A utilização de 89,2\% indica boa consolidação global, com folga de aproximadamente 65t
(11,8\%) devido à indivisibilidade das carrocerias.

\subsubsection{Modal Ferroviário}

O custo ferroviário é idêntico (R\$ 8.850,00) nas três instâncias porque:
\begin{itemize}
    \item A distribuição de fluxos S→P não muda entre instâncias
    \item Toda produção (535t) é enviada para P3
    \item As rotas utilizadas são as mais econômicas da matriz de custos
\end{itemize}

\subsection{Gargalos Identificados}

\textbf{Capacidade dos silos:} Três silos (S2, S3, S4) operam saturados em todas as
instâncias. O sistema opera com pouca folga estrutural, o que:
\begin{itemize}
    \item Impede o fechamento de qualquer silo independentemente do custo fixo
    \item Reduz flexibilidade operacional
    \item Aumenta vulnerabilidade a variações de demanda
\end{itemize}

\textbf{Restrição de carrocerias:} Duas rotas (F1→S3 e F5→S4) operam no limite de 3
carrocerias, criando gargalos locais que podem impedir expansões futuras.

\textbf{Concentração em um único porto:} A ativação exclusiva de P3 cria dependência
logística. Embora economicamente ótima, esta configuração pode ser arriscada do ponto de
vista operacional.

\section{Conclusão e Relevância do Modelo}

\subsection{Utilidade Prática}

O modelo MIP desenvolvido demonstrou ser uma ferramenta eficaz para planejamento logístico
multimodal. A capacidade de considerar simultaneamente decisões de localização (silos),
alocação (fluxos) e dimensionamento (carrocerias) confere ao modelo caráter abrangente.

A decomposição de custos revelou que o componente ferroviário domina o custo total
(69--92\%), apontando imediatamente para a necessidade de priorizar negociações neste modal.

\subsection{Apoio a Decisões Empresariais}

Com base nos resultados, um gestor poderia:

\begin{enumerate}
    \item \textbf{Priorizar negociação ferroviária:} Uma redução de 10\% nas tarifas
    ferroviárias geraria economia de R\$ 885,00 por ciclo.

    \item \textbf{Avaliar investimento em capacidade:} Expansão de 115t em qualquer dos três
    maiores silos permitiria desativar S1 em cenários de custo fixo alto.

    \item \textbf{Diversificar destinos:} A concentração em P3 cria risco operacional. Estudar
    prêmios comerciais que compensem custos logísticos adicionais para P1 ou P2.

    \item \textbf{Otimizar frota rodoviária:} Com 89,2\% de utilização, há espaço para
    ajustes. Rotas no limite (F1→S3, F5→S4) requerem atenção para expansões futuras.
\end{enumerate}

\subsection{Limitações do Modelo}

\begin{itemize}
    \item \textbf{Ausência de custos rodoviários diferenciados:} O modelo não diferencia
    distâncias entre fazendas e silos, assumindo custo uniforme por carroceria. Na realidade,
    distâncias variadas implicariam custos distintos.

    \item \textbf{Falta de dimensão temporal:} Não considera tempos de viagem, janelas de
    entrega ou o limite de 72 horas de armazenagem nos silos.

    \item \textbf{Parâmetros determinísticos:} Não modela incertezas de safra, flutuações de
    custos ou variações de demanda.

    \item \textbf{Critério único:} Minimiza apenas custos, sem considerar confiabilidade,
    tempo de trânsito ou emissões.
\end{itemize}

\subsection{Extensões Futuras}

\textbf{Recomendações:}
\begin{enumerate}
    \item Incorporar matriz de custos diferenciada por distância no trecho F→S
    \item Modelar incerteza via programação estocástica
    \item Incluir restrições temporais e programação multi-período
    \item Estender para otimização multiobjetivo (custo, tempo, emissões)
    \item Adicionar restrições de diversificação de portos para reduzir risco
\end{enumerate}

\subsection{Considerações Finais}

O trabalho demonstrou que a Programação Inteira Mista é adequada para planejamento logístico
multimodal com decisões integradas. O modelo captura decisões estratégicas (localização),
táticas (alocação) e operacionais (dimensionamento), fornecendo soluções ótimas.

A análise revelou que gargalos de capacidade dominam a estrutura da solução (todos os silos
são necessários), e que o componente ferroviário é o principal centro de custo (70--92\%).

A metodologia é aplicável a outras cadeias agroindustriais que envolvam transbordo multimodal,
decisões de localização e gestão de fluxos em redes, contribuindo para o corpo de conhecimento
em pesquisa operacional aplicada à logística.

\end{document}