\documentclass[12pt,a4paper]{article}

\usepackage[brazil]{babel}
\usepackage[utf8]{inputenc}
\usepackage[T1]{fontenc}
\usepackage{setspace}
\usepackage{geometry}
\usepackage{amsmath, amssymb}
\usepackage{graphicx}
\usepackage{booktabs}
\usepackage{multirow}

\geometry{
    left=3cm,
    right=2cm,
    top=3cm,
    bottom=2cm
}

\onehalfspacing

\begin{document}

\begin{center}
    \textbf{UNIVERSIDADE FEDERAL DO MARANHÃO} \\
    \textbf{CENTRO DE CIÊNCIAS EXATAS E TECNOLOGIA} \\[3cm]

    \textbf{DISCIPLINA:} Pesquisa Operacional \\
    \textbf{DOCENTE:} Alexandre Cesar Muniz de Oliveira \\
    \textbf{DISCENTE:} Renef Ricardo Costa da Silva \\[4cm]

    \textbf{\Large A Jornada dos Bacuris} \\[0.5cm]
    \textbf{Trabalho de Pesquisa Operacional} \\[4cm]

    São Luís -- MA \\
    2025
\end{center}

\newpage

\section{Introdução}

A otimização de sistemas logísticos com múltiplos estágios de transporte e armazenagem é
fundamental para a viabilidade econômica de cadeias produtivas complexas. Quando decisões
sobre a ativação de instalações intermediárias estão envolvidas, o problema assume caráter
estratégico e pode ser adequadamente modelado por meio da Programação Inteira Mista (MIP).

O presente trabalho aborda o Problema de Transbordo com Custos Fixos aplicado à cadeia
logística da produção de bacuri no Norte do Brasil. Este problema combina variáveis contínuas,
associadas aos fluxos de produto, e variáveis discretas, relacionadas à abertura de silos e à
quantidade de veículos utilizados, caracterizando um modelo de Programação Inteira Mista
capaz de determinar simultaneamente rotas de transporte, pontos de transbordo ativos e os
recursos logísticos necessários.

A relevância deste estudo estende-se além do caso específico do bacuri, oferecendo uma
metodologia aplicável a diversas cadeias agroindustriais que operam sob limitações de
infraestrutura e demandam soluções de transporte multimodal eficientes.

\section{Contextualização do Problema}

\subsection{O Sistema Logístico}

A cadeia logística da produção de bacuri envolve múltiplas etapas de transporte, decorrentes
das grandes distâncias entre as áreas de produção e os centros exportadores. O fluxo logístico
pode ser descrito da seguinte forma:

\begin{itemize}
    \item \textbf{Fazendas $\rightarrow$ Silos:} A produção é transportada por treminhões,
    caminhões com até três carrocerias articuladas, capazes de operar em vias não pavimentadas,
    ligando as fazendas aos silos de transbordo.

    \item \textbf{Silos (Transbordo):} Instalações temporárias que armazenam os frutos por até
    72 horas. A utilização de cada silo é opcional e condicionada ao pagamento de um custo fixo
    de ativação.

    \item \textbf{Silos $\rightarrow$ Terminal Ferroviário:} O transporte de longa distância é
    realizado por locomotivas, com custo proporcional à quantidade transportada.

    \item \textbf{Terminal $\rightarrow$ Porto:} A carga é embarcada em navios cargueiros para
    exportação aos mercados internacionais.
\end{itemize}

\subsection{Características do Problema}

O problema logístico analisado apresenta as seguintes características principais:

\begin{itemize}
    \item Múltiplos nós na rede, incluindo fazendas (origens), silos (nós intermediários),
    terminais ferroviários e porto (destino final);
    \item Utilização de múltiplos modais de transporte, como treminhões, trem e navio;
    \item Presença de custos fixos, associados à ativação dos silos e ao uso de carrocerias de
    treminhão;
    \item Capacidades limitadas nos silos e nos veículos de transporte;
    \item Necessidade de conservação de fluxo nos nós intermediários.
\end{itemize}

\subsection{Justificativa do Uso de Silos}

A utilização de silos de transbordo é essencial para a eficiência da cadeia logística, pois
permite a consolidação de cargas provenientes de múltiplas fazendas, viabiliza a integração
entre modais de transporte incompatíveis (treminhão e ferrovia), oferece flexibilidade
temporal para o escoamento da produção e contribui para a redução de custos quando
comparada ao transporte direto das fazendas até os centros exportadores.

\section{Objetivos}

\subsection{Objetivo Geral}

Desenvolver um modelo de Programação Inteira Mista que minimize o custo total da operação
logística, determinando simultaneamente os fluxos de produto, a ativação de silos de
transbordo e a quantidade de carrocerias de treminhão utilizadas.

\subsection{Objetivos Específicos}

Os objetivos específicos deste trabalho são:

\begin{itemize}
    \item Mapear o sistema logístico completo, identificando nós, arcos, capacidades e custos;
    \item Formular matematicamente o problema, definindo conjuntos, parâmetros, variáveis de
    decisão, função objetivo e restrições;
    \item Implementar computacionalmente o modelo de otimização;
    \item Criar e resolver três instâncias distintas:
    \begin{itemize}
        \item Instância A: baixo custo fixo de ativação dos silos;
        \item Instância B: alto custo fixo de ativação dos silos;
        \item Instância C: configuração de referência fornecida.
    \end{itemize}
    \item Analisar comparativamente os resultados obtidos, considerando silos ativados,
    custos totais e possíveis gargalos logísticos;
    \item Discutir a aplicabilidade prática do modelo, bem como suas limitações e possíveis
    extensões.
\end{itemize}

\section{Descrição Textual do Sistema Logístico}

\subsection{Nós da Rede}

A rede logística considerada é composta por três tipos de nós:

\textbf{Origens (Fazendas):}
\begin{itemize}
    \item $F1, F2, F3, F4, F5$: fazendas produtoras de bacuri, cada uma com capacidade de produção conhecida.
\end{itemize}

\textbf{Nós Intermediários (Silos de Transbordo):}
\begin{itemize}
    \item $S1, S2, S3, S4$: instalações temporárias de transbordo que podem ser ativadas ou não mediante o pagamento de um custo fixo.
\end{itemize}

\textbf{Destinos (Terminais Portuários):}
\begin{itemize}
    \item $P1, P2, P3$: portos exportadores que possuem demandas mínimas a serem atendidas.
\end{itemize}

\subsection{Arcos e Modais de Transporte}

\textbf{Fazenda $\rightarrow$ Silo (Modal: Treminhão)}
\begin{itemize}
    \item Arcos: $(F_i, S_j)$, para todo $i \in \{1,\dots,5\}$ e $j \in \{1,\dots,4\}$;
    \item Veículos: treminhões com configuração flexível de 1 a 3 carrocerias;
    \item Capacidade: 30 toneladas por carroceria;
    \item Custo: \$20 por carroceria utilizada (custo fixo) + \$5 por tonelada transportada.
\end{itemize}

\textbf{Silo $\rightarrow$ Porto (Modal: Ferroviário)}
\begin{itemize}
    \item Arcos: $(S_j, P_k)$, para todo $j \in \{1,\dots,4\}$ e $k \in \{1,\dots,3\}$;
    \item Custo: proporcional à quantidade transportada (em \$/tonelada), variando conforme a rota.
\end{itemize}

\subsection{Capacidades e Limites}

\begin{center}
\begin{tabular}{|c|c|c|}
\hline
\textbf{Elemento} & \textbf{Capacidade/Limite} & \textbf{Unidade} \\
\hline
Carroceria de treminhão & 30 & toneladas \\
Carrocerias por treminhão & 1 a 3 & unidades \\
Silo $S1$ & 130 & toneladas \\
Silo $S2$ & 120 & toneladas \\
Silo $S3$ & 140 & toneladas \\
Silo $S4$ & 160 & toneladas \\
\hline
\end{tabular}
\end{center}

\subsection{Justificativa Operacional}

A utilização dos silos de transbordo é justificada pelos seguintes fatores:
\begin{itemize}
    \item Consolidação de cargas provenientes de múltiplas fazendas;
    \item Interface entre modais de transporte distintos, como o rodoviário (treminhão) e o ferroviário;
    \item Possibilidade de armazenagem temporária por até 72 horas, permitindo a sincronização logística.
\end{itemize}

A flexibilidade no uso de carrocerias possibilita:
\begin{itemize}
    \item Adequação da capacidade de transporte à quantidade produzida;
    \item Um trade-off entre custo fixo e capacidade efetivamente utilizada;
    \item Decisões operacionais sobre o número de carrocerias empregadas em cada rota Fazenda $\rightarrow$ Silo.
\end{itemize}

\section{Modelagem Matemática}

Esta seção apresenta a formulação matemática do Problema de Transbordo com Custos Fixos
aplicado à cadeia logística da produção de bacuri. O modelo é formulado como um problema
de Programação Inteira Mista (MIP), combinando variáveis contínuas, inteiras e binárias.

\subsection{Conjuntos e Índices}

\begin{itemize}
    \item $F = \{F1, F2, F3, F4, F5\}$: conjunto das fazendas produtoras;
    \item $S = \{S1, S2, S3, S4\}$: conjunto dos silos de transbordo;
    \item $P = \{P1, P2, P3\}$: conjunto dos portos exportadores;
    \item $i \in F$: índice das fazendas;
    \item $j \in S$: índice dos silos;
    \item $k \in P$: índice dos portos.
\end{itemize}

\subsection{Parâmetros}

\textbf{Produção e Demanda}

\begin{itemize}
    \item $\text{prod}_i$: produção disponível na fazenda $i$ (toneladas);
    \item $\text{dem}_k$: carga mínima economicamente viável para operação do porto $k$ (toneladas).
\end{itemize}

\[
\text{prod} = \{120, 80, 95, 110, 130\}
\quad
\text{dem} = \{200, 220, 215\}
\]

\textbf{Capacidades}

\begin{itemize}
    \item $\text{cap}_j$: capacidade máxima do silo $j$ (toneladas);
    \item $\text{cap\_carr} = 30$: capacidade de uma carroceria de treminhão (toneladas).
\end{itemize}

\[
\text{cap} = \{130, 120, 140, 160\}
\]

\textbf{Custos}

\begin{itemize}
    \item $\text{cf\_silo}_j$: custo fixo de ativação do silo $j$;
    \item $\text{cf\_carr} = 20$: custo fixo por carroceria de treminhão utilizada;
    \item $c_{jk}$: custo de transporte ferroviário do silo $j$ ao porto $k$ (\$/tonelada).
\end{itemize}

\subsection{Variáveis de Decisão}

\textbf{Variáveis Contínuas (Fluxos)}

\begin{itemize}
    \item $x_{ij} \geq 0$: quantidade transportada da fazenda $i$ ao silo $j$ (toneladas);
    \item $y_{jk} \geq 0$: quantidade transportada do silo $j$ ao porto $k$ (toneladas).
\end{itemize}

\textbf{Variáveis Binárias}

\begin{itemize}
    \item $z_j \in \{0,1\}$: indica se o silo $j$ é ativado;
    \item $w_k \in \{0,1\}$: indica se o porto $k$ é ativado e opera.
\end{itemize}

\[
w_k =
\begin{cases}
1, & \text{se o porto } k \text{ opera (recebe carga)} \\
0, & \text{caso contrário}
\end{cases}
\]

\textbf{Variáveis Inteiras}

\[
t_{ij} \in \{0,1,2,3\}: \text{número de carrocerias utilizadas no transporte da fazenda } i \text{ ao silo } j
\]

\subsection{Função Objetivo}

O objetivo do modelo é minimizar o custo total da operação logística, composto pelos custos
fixos de ativação dos silos, custos fixos das carrocerias de treminhão e custos variáveis de
transporte ferroviário.

\[
\min Z =
\sum_{j \in S} \text{cf\_silo}_j \cdot z_j
+
\sum_{i \in F} \sum_{j \in S} \text{cf\_carr} \cdot t_{ij}
+
\sum_{j \in S} \sum_{k \in P} c_{jk} \cdot y_{jk}
\]

\subsection{Restrições}

\textbf{R1. Conservação de Fluxo nas Fazendas}

Toda a produção disponível em cada fazenda deve ser enviada aos silos.

\[
\sum_{j \in S} x_{ij} = \text{prod}_i
\quad \forall i \in F
\]

\textbf{R2. Conservação de Fluxo nos Silos}

Não há armazenamento permanente nos silos; todo o fluxo que entra deve sair.

\[
\sum_{i \in F} x_{ij} = \sum_{k \in P} y_{jk}
\quad \forall j \in S
\]

\textbf{R3. Ativação Condicional dos Portos}

Um porto só opera caso receba ao menos a carga mínima necessária para viabilizar a exportação.

\[
\text{dem}_k \cdot w_k
\leq
\sum_{j \in S} y_{jk}
\leq
M \cdot w_k
\quad \forall k \in P
\]

onde $M$ é uma constante suficientemente grande, definida como a soma total da produção das
fazendas.

\textbf{R4. Capacidade dos Silos}

O envio de produto a um silo só é permitido se ele estiver ativado, respeitando sua capacidade máxima.

\[
\sum_{i \in F} x_{ij}
\leq
\text{cap}_j \cdot z_j
\quad \forall j \in S
\]

\textbf{R5. Capacidade das Carrocerias}

A quantidade transportada da fazenda ao silo não pode exceder a capacidade total das carrocerias utilizadas.

\[
x_{ij}
\leq
\text{cap\_carr} \cdot t_{ij}
\quad \forall i \in F, \ j \in S
\]

\textbf{R6. Domínio das Variáveis}

\[
\begin{aligned}
x_{ij} &\geq 0 && \forall i \in F,\ j \in S \\
y_{jk} &\geq 0 && \forall j \in S,\ k \in P \\
z_j &\in \{0,1\} && \forall j \in S \\
w_k &\in \{0,1\} && \forall k \in P \\
t_{ij} &\in \{0,1,2,3\} && \forall i \in F,\ j \in S
\end{aligned}
\]

\subsection{Observações sobre o Modelo}

O modelo proposto incorpora decisões estratégicas e operacionais de forma integrada. A
restrição R3 introduz a ativação condicional dos portos, representando a necessidade de
formação de lotes mínimos economicamente viáveis para exportação. Caso essa condição não
seja atendida, o porto permanece inativo e não recebe fluxo.

A restrição R4 implementa um mecanismo do tipo \textit{big-M}, garantindo que silos não
ativados não recebam produto. A restrição R5 estabelece o vínculo entre a capacidade física
de transporte e o número de carrocerias utilizadas, criando um trade-off entre custo fixo e
viabilidade logística.

O custo de transporte no trecho Fazenda $\rightarrow$ Silo é considerado com um componente
fixo, representado pelo uso das carrocerias, e um componente variável de R\$ 5,00 por tonelada
transportada, enquanto o transporte ferroviário apresenta custo variável proporcional à
quantidade transportada com tarifas diferenciadas por rota.

\newpage

\section{Resultados e Análise Crítica}

\subsection{Ativação de Silos e Justificativas}

As três instâncias analisadas (A, B e C) utilizam os mesmos dados de produção, capacidades
e custos de transporte; a única diferença entre elas está no custo fixo de ativação dos silos
(cenários de custo fixo mais baixo, base e mais alto). Em todos os cenários, o modelo ativou
os quatro silos (S1, S2, S3 e S4), processando, respectivamente, 115 t, 120 t, 140 t e 160 t.

A escolha por manter todos os silos abertos não é apenas uma preferência econômica, mas uma
consequência estrutural do sistema. A produção total é 535 t e, se um silo fosse fechado, a
capacidade somada dos três restantes não seria suficiente para absorver toda a produção dentro
dos limites individuais de cada silo. Em particular, os três maiores silos (S4, S3 e S1)
totalizam $160 + 140 + 130 = 420$ t, o que impõe a ativação do quarto silo para viabilizar
o escoamento completo.

\subsubsection{Justificativas por Silo}

\textbf{S4 (160 t processados):} Aparece em todas as instâncias por ser o silo de maior
capacidade e opera saturado em todos os cenários. O trade-off associado é que, mesmo com
custo fixo alto, sua capacidade e participação nas rotas ferroviárias mais econômicas reduzem
a necessidade de redistribuir carga para rotas mais caras.

\textbf{S3 (140 t processados):} Também é saturado por combinar alta capacidade e custos
ferroviários atrativos (especialmente para P3 com custo de R\$ 10/t). A presença constante
indica robustez: mesmo com variação relevante de custo fixo, a estrutura de custos variáveis
e o gargalo de capacidade fazem com que o silo permaneça indispensável.

\textbf{S2 (120 t processados):} Opera no limite de capacidade, funcionando como complemento
obrigatório para fechar o balanço de capacidade. Sua ativação não é apenas decisão de custo
fixo; é necessária para acomodar o volume total sem violar as capacidades dos demais silos.

\textbf{S1 (115 t processados):} Atua como ``amortecedor'' do sistema. Embora não esteja
saturado (capacidade 130 t), ele é necessário para absorver o restante de produção após o
preenchimento dos silos mais atrativos/capacitivos, preservando a viabilidade e evitando
exceder limites operacionais.

\subsection{Custos Totais e Decomposição}

A Tabela \ref{tab:custos} apresenta o custo total em cada instância e a decomposição por componente.

\begin{table}[h]
\centering
\caption{Decomposição de custos por instância}
\label{tab:custos}
\begin{tabular}{lccc}
\toprule
\textbf{Componente (R\$)} & \textbf{Instância A} & \textbf{Instância C} & \textbf{Instância B} \\
\midrule
Transporte Fazenda$\rightarrow$Silo & 2.675,00 & 2.675,00 & 2.675,00 \\
Custo fixo dos silos & 395,00 & 1.950,00 & 3.550,00 \\
Transporte Silo$\rightarrow$Porto & 8.850,00 & 8.850,00 & 8.850,00 \\
Custo das carrocerias & 400,00 & 400,00 & 400,00 \\
\midrule
\textbf{Custo total} & \textbf{9.645,00} & \textbf{11.200,00} & \textbf{12.800,00} \\
\bottomrule
\end{tabular}
\end{table}

\subsubsection{Análise Percentual}

Em todas as instâncias, o transporte ferroviário (Silo$\rightarrow$Porto) é a maior parcela
do custo total, pois concentra o maior custo variável por tonelada e reflete a matriz
diferenciada de custos por rota. Na Instância A, a ferrovia representa aproximadamente
$\frac{8.850}{9.645} \approx 91,8\%$ do custo total, enquanto o custo fixo representa cerca de
$\frac{395}{9.645} \approx 4,1\%$. Na Instância B, o custo fixo cresce para cerca de
$\frac{3.550}{12.800} \approx 27,7\%$, reduzindo a participação relativa da ferrovia para
$\approx 69,1\%$, mesmo sem alteração nas rotas ferroviárias.

O transporte rodoviário (treminhão) representa aproximadamente 27,7\% do custo na Instância A,
20,9\% na Instância B, mantendo-se constante em valor absoluto (R\$ 2.675,00) devido à
linearidade do custo unitário de R\$ 5,00/t e à produção total fixa de 535 t.

\subsubsection{Comparação Relativa}

A Instância A é a mais barata porque reduz o componente fixo sem alterar a operação física
(volumes e saturações) ditada pelas restrições. A Instância C é a referência do cenário base.
A Instância B é a mais cara: como o sistema já está no limite de capacidade, o aumento dos
custos fixos não leva ao fechamento de silos (decisão estrutural), sendo absorvido diretamente
no custo total. A diferença entre a instância mais cara e a mais barata é de R\$ 3.155,00,
representando um aumento de 32,7\%.

\subsection{Análise Comparativa das Instâncias}

\subsubsection{Número de Silos e Sensibilidade aos Custos Fixos}

O número de silos ativados é o mesmo nas três instâncias (4 silos), evidenciando que a decisão
de abertura é pouco sensível ao custo fixo no patamar atual, pois existe um gargalo de
capacidade. Isso produz um comportamento típico de ``efeito limiar'': enquanto o sistema
estiver próximo ao limite, mudanças no custo fixo mudam o valor ótimo, mas não mudam a
estrutura (quais silos abrem/fecham).

A Tabela \ref{tab:comparativa} apresenta os principais indicadores comparativos entre as três instâncias.

\begin{table}[h]
\centering
\caption{Comparação das três instâncias (indicadores principais)}
\label{tab:comparativa}
\begin{tabular}{lccc}
\toprule
\textbf{Indicador} & \textbf{Instância A} & \textbf{Instância C} & \textbf{Instância B} \\
\midrule
Custo total (R\$) & 9.645,00 & 11.200,00 & 12.800,00 \\
Silos ativados (quant.) & 4 & 4 & 4 \\
Custo fixo total (R\$) & 395,00 & 1.950,00 & 3.550,00 \\
Porto mais atendido (t) & P3 (535) & P3 (535) & P3 (535) \\
Carrocerias utilizadas & 20 & 20 & 20 \\
Utilização agregada (\%) & 89,2 & 89,2 & 89,2 \\
\bottomrule
\end{tabular}
\end{table}

\subsubsection{Distribuição dos Fluxos e Padrões Observados}

Os fluxos Silo$\rightarrow$Porto são idênticos nas três instâncias, indicando uma solução
robusta no nível ferroviário. Em todas as instâncias, o porto P3 recebe toda a produção
(535 t), saturando sua capacidade de 215 t. Este padrão mostra uma priorização econômica
clara de P3, que apresenta os menores custos ferroviários entre os silos e o destino final.

A matriz de custos ferroviários evidencia que as rotas para P3 são sistematicamente mais
econômicas:
\begin{itemize}
    \item $S4 \rightarrow P3$: R\$ 10/t
    \item $S3 \rightarrow P3$: R\$ 10/t
    \item $S2 \rightarrow P3$: R\$ 20/t
    \item $S1 \rightarrow P3$: R\$ 30/t
\end{itemize}

Essa estrutura de custos explica por que, mesmo com P3 operando acima de sua capacidade
nominal de 215 t, o modelo prioriza esse destino, pois os custos marginais de transporte são
significativamente menores que as alternativas para P1 e P2.

\subsection{Impacto dos Modais de Transporte}

\subsubsection{Treminhão (Fazenda$\rightarrow$Silo)}

O custo total do trecho rodoviário é R\$ 2.675,00 em todas as instâncias, pois a produção
total é constante e o custo por tonelada no modal rodoviário é linear (R\$ 5,00/t). O número
total de carrocerias utilizadas é 20, resultando em capacidade contratada total de
$20 \times 30 = 600$ t. Assim, a utilização média agregada do sistema rodoviário é
aproximadamente $\frac{535}{600} \approx 89,2\%$, indicando boa consolidação global, mas com
inevitável ``folga'' por indivisibilidade (variáveis inteiras).

A restrição de no máximo 3 carrocerias por rota limita a concentração de fluxo em uma única
rota fazenda$\rightarrow$silo e força a distribuição em múltiplas rotas quando necessário.
Na prática, isso representa limites de frota/viagens e pode aumentar a complexidade
operacional (mais pontos de carga/descarga e coordenação). Nas soluções obtidas, duas rotas
operam no limite máximo: $F1 \rightarrow S3$ (90 t com 3 carrocerias) e
$F5 \rightarrow S4$ (80 t com 3 carrocerias), indicando que a restrição operacional está
ativa nesses pontos.

\subsubsection{Ferrovia (Silo$\rightarrow$Porto)}

O custo ferroviário total é R\$ 8.850,00 em todas as instâncias e domina o custo global,
refletindo a matriz de custos diferenciados por rota. O porto P3 é priorizado e saturado,
comportamento coerente quando se busca minimizar custos de transporte em um sistema com
custos variáveis diferenciados.

\subsubsection{Comparação Modal}

O modal ferroviário é mais caro em termos absolutos porque concentra a maior parte do custo
por tonelada e o maior volume agregado, enquanto o modal rodoviário é modelado com custo
linear uniforme e menor peso total. A diferença entre os custos ferroviário e rodoviário é
de R\$ 6.175,00, evidenciando que o transporte de longa distância é o principal componente
de custo do sistema.

\subsection{Gargalos Logísticos Identificados}

\subsubsection{Capacidade dos Silos}

Há saturação de capacidade em três silos: S4 (160/160 t), S3 (140/140 t) e S2 (120/120 t),
com S1 absorvendo o restante (115/130 t). Isso explica por que todos os silos aparecem em
todas as instâncias e por que o modelo não consegue reduzir o número de silos abertos: a rede
opera com pouca folga estrutural.

Esta situação crítica indica que o sistema está operando próximo ao seu limite técnico,
o que reduz a flexibilidade estratégica e aumenta a vulnerabilidade a variações de demanda
ou falhas operacionais. Qualquer aumento na produção além das 535 t atuais exigiria
necessariamente a expansão da capacidade de pelo menos um silo.

\subsubsection{Restrição de Carrocerias}

Existem rotas usando 3 carrocerias (limite operacional), indicando que a restrição
operacional é ativa e pode criar gargalos locais. Mesmo quando o custo por tonelada é
linear, a integralidade das carrocerias (0--3) pode forçar fracionamento de fluxos e criar
pequenas ineficiências por capacidade ociosa. As rotas $F1 \rightarrow S3$ e
$F5 \rightarrow S4$ operam no limite, o que sugere que eventuais aumentos de produção
nessas fazendas encontrariam restrições imediatas de transporte.

\subsubsection{Desequilíbrio Oferta/Demanda}

A produção total (535 t) é menor que o somatório dos limites dos portos (635 t). No modelo
utilizado, os portos operam como limites superiores e não como metas mínimas, o que permite
distribuir a oferta escassa priorizando os destinos com melhor custo e deixando os portos
menos atrativos ociosos. Na prática, apenas P3 é utilizado em todas as instâncias, indicando
uma forte concentração de fluxo em função dos custos ferroviários mais baixos para este
destino.

\subsubsection{Custos Altos Específicos e Flexibilidade Limitada}

Como a ativação dos silos é binária, o sistema perde flexibilidade estratégica quando está
no limite de capacidade: aumentos de custo fixo (Instância B) não mudam a estrutura (quais
silos abrem), apenas encarecem a solução. Esse resultado sugere que melhorias de capacidade
(expansão de um silo ou aumento de eficiência) podem gerar mais retorno estratégico do que
apenas renegociar custo fixo, pois devolvem ao modelo a possibilidade real de desligar
instalações quando economicamente vantajoso.

\section{Conclusão e Relevância do Modelo}

\subsection{Utilidade Prática}

O modelo de Programação Inteira Mista desenvolvido neste trabalho demonstra ser uma
ferramenta útil e aplicável para o planejamento de operações logísticas multimodais com
decisões estratégicas integradas. A capacidade de considerar simultaneamente decisões de
localização (quais silos ativar), alocação (como distribuir fluxos) e dimensionamento
(quantas carrocerias contratar) confere ao modelo um caráter abrangente que reflete a
complexidade real de cadeias agroindustriais.

O modelo é útil para planejar escoamento multimodal com decisões estratégicas (ativação de
silos) e decisões táticas/operacionais (alocação de fluxo e contratação de carrocerias). A
decomposição de custos mostra claramente onde estão os maiores ``centros de custo'' e permite
testar cenários de contrato (A, B e C), apoiando o planejamento de orçamento e a negociação
com fornecedores de serviços logísticos.

A análise dos resultados revela que o componente ferroviário domina o custo total em todas
as instâncias (variando de 69,1\% a 91,8\% dependendo do cenário), o que aponta
imediatamente para a necessidade de priorizar a gestão e negociação deste modal. Além disso,
a identificação de gargalos estruturais (três silos operando no limite de capacidade)
fornece informações valiosas para decisões de investimento em infraestrutura.

A robustez da solução ferroviária, que permanece idêntica nas três instâncias apesar das
variações significativas nos custos fixos dos silos (de R\$ 395,00 na Instância A para
R\$ 3.550,00 na Instância B), indica que as decisões de transporte de longa distância são
determinadas principalmente pela estrutura de custos variáveis e capacidades, e não pelos
custos fixos de transbordo. Essa característica confere previsibilidade ao sistema e facilita
o planejamento de médio e longo prazo.

\subsection{Apoio a Decisões Empresariais}

Com base nos resultados obtidos, um gestor logístico poderia tomar decisões fundamentadas
em diversas frentes estratégicas e operacionais:

\textbf{Negociação de Contratos de Transporte:} Priorizar a negociação de tarifas e
contratos ligados ao modal ferroviário, pois ele domina o custo total. Uma redução de
apenas 10\% nas tarifas ferroviárias resultaria em economia de aproximadamente R\$ 885,00
por ciclo operacional, valor significativamente superior ao que seria obtido com reduções
equivalentes em outros componentes. A concentração de fluxo no porto P3 sugere que contratos
de longo prazo ou parcerias estratégicas com operadores ferroviários nesta rota específica
poderiam gerar economias substanciais.

\textbf{Investimento em Capacidade:} Avaliar investimento em capacidade de transbordo (por
exemplo, expansão de um silo) para criar folga operacional e permitir fechamento seletivo
de silos em cenários de custo fixo alto. A análise revelou que a capacidade combinada dos
três maiores silos (420 t) é insuficiente para a produção total (535 t), forçando a
manutenção do quarto silo independentemente do custo fixo. Um aumento de capacidade de
apenas 115 t em qualquer um dos três maiores silos criaria a possibilidade de desativar
S1 quando seus custos fixos fossem proibitivos, potencialmente economizando até R\$ 900,00
na Instância B.

\textbf{Estratégia Comercial de Atendimento:} Rever estratégia comercial de atendimento
de portos. Com oferta limitada (535 t) e demanda total potencial de 635 t, faz sentido
priorizar os portos que minimizam custo logístico e usar portos menos atrativos apenas quando
necessário ou quando oferecerem prêmios comerciais que compensem os custos adicionais de
transporte. A solução atual, que concentra todo o fluxo em P3, pode não ser a mais adequada
do ponto de vista comercial se houver diferenciais de preço entre os portos ou penalidades
por não atendimento de contratos.

\textbf{Gestão de Frota Rodoviária:} Otimizar a utilização de carrocerias, que apresenta
taxa agregada de 89,2\%. Embora esta seja uma taxa relativamente alta, a identificação de
rotas operando no limite máximo (3 carrocerias) sugere pontos de atenção para expansão
futura. A existência de capacidade ociosa de aproximadamente 65 t (600 t contratadas vs
535 t transportadas) indica oportunidade para redimensionamento da frota ou uso compartilhado
de recursos em períodos de baixa demanda.

\textbf{Planejamento de Contingência:} Desenvolver planos de contingência para os três
silos que operam saturados (S2, S3 e S4). Como o sistema está no limite de capacidade,
qualquer interrupção operacional em um desses silos comprometeria toda a cadeia logística.
Investimentos em redundância ou em capacidade de armazenagem temporária adicional podem
ser justificados pela redução de risco operacional.

\textbf{Análise de Sensibilidade para Precificação:} Utilizar o modelo para análises de
sensibilidade que apoiem decisões de precificação e aceitação de contratos. Por exemplo,
se um cliente solicitar entrega em P1 ou P2 em vez de P3, o modelo pode quantificar
rapidamente o custo adicional dessa exigência, permitindo ajustes de preço ou negociação
de termos contratuais que reflitam os custos logísticos reais.

\subsection{Limitações e Perspectivas Futuras}

Embora o modelo desenvolvido seja robusto e aplicável, algumas limitações devem ser
reconhecidas, abrindo caminho para extensões e melhorias futuras:

\subsubsection{Limitações do Modelo Atual}

\textbf{Simplificação de Custos Rodoviários:} O modelo assume custo linear do transporte
rodoviário por tonelada (R\$ 5,00/t), sem diferenciar rotas fazenda$\rightarrow$silo. Na
realidade, diferentes fazendas podem estar a distâncias variadas dos silos, resultando em
custos diferenciados. Essa simplificação pode gerar múltiplas soluções equivalentes na
distribuição das fazendas aos silos, tornando a solução ótima não única do ponto de vista
prático. Uma matriz de custos ou distâncias específica para cada arco
fazenda$\rightarrow$silo aumentaria o realismo e a aplicabilidade do modelo.

\textbf{Ausência de Dimensão Temporal:} O modelo não considera explicitamente tempos de
viagem, janelas de entrega, filas operacionais, ou a dinâmica temporal da chegada de
produtos aos silos. Em operações reais, a sincronização temporal entre os modais é crucial,
especialmente considerando o limite de 72 horas de armazenagem nos silos. A incorporação
de restrições temporais e janelas de operação tornaria o modelo mais aderente à realidade
operacional.

\textbf{Determinismo dos Parâmetros:} Todos os parâmetros são tratados como valores
determinísticos conhecidos a priori. Na prática, a produção das fazendas pode variar devido
a fatores climáticos, fitossanitários ou sazonais; os custos de transporte podem flutuar
em função do preço de combustíveis; e a disponibilidade de capacidade nos portos pode ser
afetada por congestionamentos. A ausência de tratamento de incerteza limita a robustez das
decisões em ambientes voláteis.

\textbf{Critério de Otimização Único:} O modelo considera exclusivamente a minimização de
custos, não incorporando outros objetivos relevantes como confiabilidade de entrega, tempo
de trânsito, emissões de carbono, ou distribuição de risco entre fornecedores. Em contextos
empresariais reais, frequentemente há trade-offs entre múltiplos objetivos que precisam ser
balanceados.

\textbf{Limitações Operacionais Simplificadas:} Aspectos operacionais como quebras de
equipamentos, manutenção preventiva, disponibilidade limitada de mão de obra, ou restrições
regulatórias não são modelados. Da mesma forma, o modelo não captura economias de escala
ou descontos por volume que poderiam existir nas tarifas de transporte.

\subsubsection{Recomendações para Melhorias Futuras}

Com base nas limitações identificadas, recomenda-se as seguintes extensões para trabalhos
futuros:

\textbf{Incorporação de Matriz de Custos Diferenciada:} Incluir uma matriz de custos (ou
distâncias) específica para cada arco fazenda$\rightarrow$silo, capturando a heterogeneidade
geográfica e econômica das rotas rodoviárias. Isso tornaria as soluções mais realistas e
únicas, eliminando a ambiguidade na alocação de fazendas a silos.

\textbf{Modelagem sob Incerteza:} Incorporar incerteza paramétrica através de programação
estocástica ou otimização robusta. Cenários probabilísticos de produção (safra alta/média/
baixa), custos de transporte (preços de combustível) e disponibilidade de capacidade
portuária poderiam ser modelados explicitamente. Alternativamente, abordagens de otimização
robusta poderiam garantir soluções que permaneçam viáveis sob variações paramétricas dentro
de limites especificados.

\textbf{Modelagem Temporal e Dinâmica:} Estender o modelo para um horizonte de planejamento
multi-período, incorporando decisões de programação de entregas ao longo do tempo. Isso
permitiria capturar a dinâmica de chegada de produtos, restrições de janelas temporais,
e o limite de 72 horas de armazenagem nos silos. Um modelo de programação inteira mista
dinâmica poderia otimizar não apenas onde, mas também quando realizar cada operação de
transporte.

\textbf{Metas Mínimas e Penalidades:} Modelar demandas dos portos como metas mínimas com
penalidades por não atendimento quando a demanda for obrigatória ou contratual. Isso
refletiria melhor situações em que há compromissos comerciais firmes que devem ser
respeitados, mesmo que a um custo adicional. A formulação poderia incluir variáveis de
escassez com custos de penalidade associados.

\textbf{Otimização Multiobjetivo:} Desenvolver uma versão multiobjetivo do modelo que
considere simultaneamente minimização de custos, minimização de tempo de trânsito,
minimização de emissões de carbono, e maximização de confiabilidade. Técnicas como programação
por metas, método do $\varepsilon$-restrito, ou algoritmos evolutivos multiobjetivo poderiam
gerar um conjunto de soluções Pareto-ótimas, permitindo ao gestor escolher o melhor
trade-off conforme as prioridades estratégicas da organização.

\textbf{Expansão para Rede Completa:} Estender o modelo para incluir explicitamente o trecho
terminal ferroviário$\rightarrow$porto, atualmente considerado de forma implícita. Isso
permitiria analisar decisões sobre transbordo em terminais ferroviários e eventuais
consolidações antes do embarque nos navios.

\textbf{Integração com Sistemas de Informação:} Desenvolver interfaces de integração com
sistemas ERP, TMS (Transportation Management Systems) e sistemas de rastreamento em tempo
real. A parametrização automática do modelo a partir de dados operacionais atualizados
permitiria seu uso regular como ferramenta de apoio à decisão, não apenas como estudo
pontual.

\subsection{Considerações Finais}

O presente trabalho demonstrou que a Programação Inteira Mista é uma abordagem adequada e
eficaz para o planejamento logístico integrado de cadeias agroindustriais multimodais. O
modelo desenvolvido para o problema de transbordo de bacuri captura as principais decisões
estratégicas, táticas e operacionais do sistema, fornecendo soluções ótimas que minimizam
custos totais enquanto respeitam todas as restrições físicas e operacionais.

A análise comparativa das três instâncias revelou insights importantes sobre a estrutura
do problema, particularmente o papel crítico dos gargalos de capacidade na determinação da
configuração ótima de silos e a dominância dos custos ferroviários na composição do custo
total. Esses resultados têm implicações diretas para decisões empresariais, desde a
priorização de negociações contratuais até investimentos em expansão de capacidade.

As limitações identificadas e as extensões propostas não diminuem o valor do modelo atual,
mas apontam caminhos promissores para o desenvolvimento de ferramentas ainda mais completas
e aderentes à complexidade operacional real. A metodologia apresentada é suficientemente
flexível para acomodar essas extensões, constituindo uma base sólida para trabalhos futuros
em otimização de cadeias logísticas agroindustriais.

Por fim, cabe ressaltar que, além do caso específico do bacuri, a estrutura metodológica
aqui desenvolvida é aplicável a uma ampla gama de problemas logísticos que envolvam
transbordo multimodal, decisões de localização, e gestão integrada de fluxos em redes. Dessa
forma, o trabalho contribui não apenas para a otimização de uma cadeia específica, mas
também para o corpo de conhecimento em pesquisa operacional aplicada à logística
agroindustrial.

\end{document}